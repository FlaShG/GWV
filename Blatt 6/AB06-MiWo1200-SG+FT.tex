\documentclass[a4paper,10pt]{article}
\usepackage[utf8]{inputenc}
\usepackage[ngerman]{babel}
\usepackage{amsmath}
%opening
\title{GWV Aufgabenblatt 6}
\author{Sascha Graeff \& Felix Templin}

\begin{document}

\maketitle

\section*{Aufgabe 1.1}
Mit $c$ ist im Folgenden immer der Übertrag aus den niedrigstelligeren Additionen und ist entsprechend eine Variable mit dem Wertebereich $\{0,1\}$. $c*10$ ist der Übertrag aus der aktuellen Addition und entsprechend im Wertebereich $\{0,10\}$.
\begin{itemize}
  \item $D + E = Y + (c*10)$
  \item $N + R + c = E$
  \item $S + M + c = MO$
\end{itemize}

\end{document}
