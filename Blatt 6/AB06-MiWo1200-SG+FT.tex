\documentclass[a4paper,10pt]{article}
\usepackage[utf8]{inputenc}
\usepackage[ngerman]{babel}
\usepackage{amsmath}
%opening
\title{GWV Aufgabenblatt 6}
\author{Sascha Graeff \& Felix Templin}

\begin{document}

\maketitle

\section*{Aufgabe 1.1}
Mit $c$ ist im Folgenden immer der Übertrag aus den niedrigstelligeren Additionen.
\begin{itemize}
  \item $S + M + c = MO$ \\
          Hier ist die höchststellige Addtion betrachtet. c ist hier der Übertrag aus den niedrigstelligeren Additionen, der 1 nicht überschreiten kann.
          Da die Summe zweier einstelliger Zahlen nicht über 18 (mit Übertrag 19) sein kann, ist die erste Ziffer einer zweistelligen Zahl, die die Summer zweier einstelliger Zahlen ist, immer 1. \\
          $\Rightarrow M = 1$
\end{itemize}

\end{document}
