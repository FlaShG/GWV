\documentclass[a4paper,10pt]{article}
\usepackage[utf8]{inputenc}
\usepackage[ngerman]{babel}
\usepackage{amsmath}
%opening
\title{GWV Aufgabenblatt 6}
\author{Sascha Graeff \& Felix Templin}

\begin{document}

\maketitle

\section*{Aufgabe 1.1}
\textbf{(a)}
$c$ ist im Folgenden die Übertragsfunktion. $c(n) = 1 \Leftrightarrow n \geq 10$, sonst $0$.
 \begin{itemize}
  \item $D + E = Y + (c(D+E) * 10)$
  \item $N + R + c(D+E) = E + (c(N + R + c(D+E)) * 10)$
  \item $E + O + c(N + R + c(D+E)) = N + (c(E + O + c(N + R + c(D+E))) * 10)$
  \item $S + M + c(E + O + c(N + R + c(D+E))) = MO$
 \end{itemize}
\textbf{(b)}
\begin{itemize}
  \item Erster Versuch: Trial\&Error, also Backtracking. Man kommt nicht besonders weit damit, da man als Mensch nur sehr langsam ausprobieren kann. Außerdem ist es schwierig, sich zu merken, was man schon ausprobiert hat.
  \item Zweiter Versuch: Alle Anfangs-, Mittel- und Endbuchstaben aufschreiben. Das Wort in der linken Spalte darf nur aus Anfangsbuchstaben bestehen, da sonst in den kreuzenden Zeilen Probleme auftreten.
          Entsprechendes für die zweite und die dritte Spalte, und dann übertragen auch auf die Zeilen.
          Mit der eingeschränkten Wortzahl wird dann wieder per Backtracking ausprobiert.\\
          Unter Zuhilfenahme weiterer logischer Ansätze, wie die Spiegelbarkeit des Feldes entlang seiner Diagonalen, führte dieser Ansatz, nachdem die erste Phase abgeschlossen war, innerhalb weniger Minuten zu einem Ergebnis:

\begin{tabular}{ | l | l | l | }
\hline
b & e & e \\ \hline
o & a & f \\ \hline
a & r & t \\ \hline
\end{tabular}
\end{itemize}
\textbf{(c)}

\end{document}
