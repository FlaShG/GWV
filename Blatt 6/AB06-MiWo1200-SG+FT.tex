\documentclass[a4paper,10pt]{article}
\usepackage[utf8]{inputenc}
\usepackage[ngerman]{babel}
\usepackage{amsmath}
%opening
\title{GWV Aufgabenblatt 6}
\author{Sascha Graeff \& Felix Templin}

\begin{document}

\maketitle

\section*{Aufgabe 1.1}
\textbf{(a)}
$c$ ist im Folgenden die Übertragsfunktion. $c(n) = 1 \Leftrightarrow n \geq 10$, sonst $0$.
 \begin{itemize}
  \item $D + E = Y + (c(D+E) * 10)$
  \item $N + R + c(D+E) = E + (c(N + R + c(D+E)) * 10)$
  \item $E + O + c(N + R + c(D+E)) = N + (c(E + O + c(N + R + c(D+E))) * 10)$
  \item $S + M + c(E + O + c(N + R + c(D+E))) = MO$
 \end{itemize}
\textbf{(b)}
\begin{itemize}
  \item Erster Versuch: Trial\&Error, also Backtracking. Man kommt nicht besonders weit damit, da man als Mensch nur sehr langsam ausprobieren kann. Außerdem ist es schwierig, sich zu merken, was man schon ausprobiert hat.
  \item Zweiter Versuch: Alle Anfangs-, Mittel- und Endbuchstaben aufschreiben. Das Wort in der linken Spalte darf nur aus Anfangsbuchstaben bestehen, da sonst in den kreuzenden Zeilen Probleme auftreten.
          Entsprechendes für die zweite und die dritte Spalte, und dann übertragen auch auf die Zeilen.
          Mit der eingeschränkten Wortzahl wird dann wieder per Backtracking ausprobiert.\\
          Unter Zuhilfenahme weiterer logischer Ansätze, wie die Spiegelbarkeit des Feldes entlang seiner Diagonalen, führte dieser Ansatz, nachdem die erste Phase abgeschlossen war, innerhalb weniger Minuten zu einem Ergebnis:

\begin{tabular}{ | l | l | l | }
\hline
b & e & e \\ \hline
o & a & f \\ \hline
a & r & t \\ \hline
\end{tabular}
\end{itemize}
\newpage
\noindent \textbf{(c) }
Wir ordnen den neun Feldern die Großbuchstaben A bis I zu.
Domain consistency: Bei den Feldern, die den ersten Buchstaben eines Wortes halten können (also A,B,C,D,G) werden alle Buchstaben gestrichen,
die nicht der erste Buchstabe eines der Wörter sein können.\\
Daraus resultiert: 
A, B, C, D, G = \{a, b, e, f, l, o, r, t\}
Danach werden bei allen Feldern, die einen zweiten Buchstaben der gegebenen Wörter halten können (also B,D,E,F,H) alle Buchstaben gestrichen, die nicht an zweiter 
Stelle stehen können.
Daraus resultiert: \\
B, D = \{a, e, f, o, r\}\\
E, F, H = \{a, d, e, f, g, i, n, o, p, r, s, u, w, y\}\\
Danach werden bei allen Feldern, die einen dritten Buchstaben der gegebenen Wörter halten können (also C,F,G,H,I) alle Buchstaben gestrichen, die nicht an dritter 
Stelle stehen können.
Daraus resultiert:\\
C, G = \{a, e, f, l, o, r, t\}\\
F, H = \{a, e, f, o, r\}\\
I =    \{a, d, e, f, h, k, l, m, n, o, r, t\}\\
Damit ist domain cositency hergestellt, die Möglichen Belegungen der Felder sind:\\
A = \{a, b, e, f, l, o, r, t\}\\
B, D = \{a, e, f, o, r\}\\
C, G = \{a, e, f, l, o, r, t\}\\
F, H = \{a, e, f, o, r\}\\
I    = \{a, d, e, f, h, k, l, m, n, o, r, t\}\\
Nun können wir damit beginnen, constraints zu berücksichtigen, die mehr als ein Feld betreffen.\\
Die constraints müssen so konstruiert sein, dass zum einen nur Wörter verwendet werden, die auch in der Wörtermenge vorkommen, 
zum anderen dürfen sich die Buchstaben in den drei in einer Zeile oder Spalte liegenden Felder nicht 
widersprechen. W sei hier die Menge der möglichen Wörter\\
So kommt je ein constraint mit den Skopus \{A,B,C\}, \{A,D,G\}, \{B,E,H\}, \{C,F,I\}, \{D,E,F\}, \{G,H,I\} zustande.
Das constraint besteht darin, dass die drei Elemente von links nach rechts oder oben nach unten verkettet ein Wort aus der Wortmenge 
W ergeben müssen, zum Beispiel: A + B + C = Element aus W (b+e+e = bee).\\
Auf diese weise ist jedes Feld an zwei constraints gebunden, die sich auf mehr als ein Feld beziehen.
Für die Auswertung dieser constraints ist es notwendig, dass wir ein Wort wählen und testen, ob alle Contraints erfüllt werden, da 
wir davon ausgehen müssen, dass es mehr als eine richtige Lösung gibt.
Also wählen wir das Wort ``bee'' für den Skopus \{A,B,C\}.
Damit schränken wir die möglichen Wörter für die Tripel \{A,D,G\} auf \{bad, bag, bat, boa\} und das Tripel \{B,E,H\} und \{C,H,I\} auf 
\{ear, eel, eft\} ein. Da wir nur ein Wort haben, das mit ``o'' beginnt, wählen wir nun für \{A,D,G\} das Wort ``boa'', damit wir ``o'' als Anfangsbuchstaben 
verwenden können. Daraus ergibt sich, dass für \{D,E,F\} nur noch ``oaf'' in Frage kommt. Diese Schritte lassen bei 
\{B,E,H\} nur noch ``ear'' und bei \{C,F,I\} nur noch ``eft'' zu. Im letzten Schritt müssen wir nur noch ablesen, dass das einzig mögliche Wort für
\{G,H,I\} ``art'' lautet. Damit ist die oben dargestellte Lösung hergeleitet.
\end{document}
