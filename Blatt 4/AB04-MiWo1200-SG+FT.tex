\documentclass[a4paper,10pt]{article}
\usepackage[utf8]{inputenc}
\usepackage[ngerman]{babel}
\usepackage{graphics,graphicx}
\usepackage{pstricks,pst-node,pst-tree}
%opening
\title{GWV Aufgabenblatt 4}
\author{Sascha Graef \& Felix Templin}

\begin{document}

\maketitle



\section*{Aufgabe 1}
Unsere Heuristik verwendet eine Messung des Restabstandes zum Ziel über die Manhattan-Distanz sowie eine
Kostenberechnung, die die Kosten des bisher zurückgelegten Pfades berücksichtigt.
Wir haben uns für die Manhattan-Distanz entschieden, da wir so eine einfache Berechnungsgrundlage ohne float-Werte haben.
\section*{Aufgabe 2}
Unsere Suche versucht zunächst, einen Pfad zum Ziel zu finden und terminiert, wenn dies unmöglich ist.
Dies wird dadurch sichergestellt, dass bereits besuchte Felder markiert werden.
\section*{Aufgabe 3}
Die Heuristik wird dadurch angepasst, dass, wenn ein Teleporterpunkt in der Frontier auftaucht, der Abstand zum Ziel nicht von diesem
Punkt, sondern von seinem Gegenstück (in der Implementation Buddy genannt) aus berechnet wird.
\section*{Aufgabe 4}
\textbf{Tiefensuche:} Bei der Tiefensuche wächst der Zeitaufwand exponentiell mit der Größe des zu druchsuchenden Bereiches, während der Platzaufwand linear wächst. 
Der Frontier werden immer nur die Nachfolger des momentanen Knotens geladen, daher wächst der Speicherbedarf linear.\\
\textbf{Breitensuche:} Der Zeitaufwand der Breitensuche entspricht der der Tiefensuche , während der Platzaufwand höher ist.
Das kommt daher, dass wir eine Queue verwenden und keinen Stack wie bei der Tiefensuche. So haben wir mehr Elemente in der Frontier.\\
\textbf{Suche mit A*:} Der Zeitaufwand ist in diesem Beispiel exponentiell. 

\end{document}
